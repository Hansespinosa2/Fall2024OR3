\section{Monday 09/30/2024}
\subsection{Review}
\subsubsection{Convex functions and sets}
\begin{equation}
  \frac{1}{2}x^T Q x + c^T x
\end{equation}
is the quadratic form and this function is convex as long as $Q$ is positive semidefinite.

\framebox[\linewidth]{
    \begin{minipage}{\dimexpr\linewidth-2\fboxsep-2\fboxrule\relax}
        A function $h: \mathbb{R}^n \to \mathbb{R}$ is said to be convex if and only if it satisfies 
        \begin{equation}
          \lambda h(x_1) + (1-\lambda)h(x_2) \geq h(\lambda x_1 + (1-\lambda)x_2)
        \end{equation}  
        for any $\lambda \in [0,1]$ and any $x_1, x_2 \in \textbf{dom } h$ 
    \end{minipage}
}
\\ \\ 
\framebox[\linewidth]{
    \begin{minipage}{\dimexpr\linewidth-2\fboxsep-2\fboxrule\relax}
        A set $\Omega \subseteq \mathbb{R}^n $ is said to be convex if and only if it satisfies 
        \begin{equation}
          \lambda x_1 + (1-\lambda)x_2 \in \Omega
        \end{equation}  
        for any $\lambda \in [0,1]$ and any $x_1, x_2 \in \Omega$ 
    \end{minipage}
}


\subsection{Convexity}
\subsubsection{Properties of a convex function}
For a twice-differential function to be convex, it's Hessian matrix must be positive semidefinite. One definition for positive semidefinite is that all eigenvalues must be non-negative, meaning that the stretch of the matrix is positive or none in the direction of each eigenvector. Below we define what the hessian of a function $f$ looks like.
\begin{align}
  \nabla^2 f(x) = 
  \begin{bmatrix}
     \frac{d^2 f(x)}{d x_1^2 } \frac{d^2 f(x)}{d x_1 x_2 } \dots \frac{d^2 f(x)}{d x_1 x_n} \\
     \frac{d^2 f(x)}{d x_2 x_1 } \frac{d^2 f(x)}{d x_2^2} \dots \frac{d^2 f(x)}{d x_2 x_n} \\
     \vdots\\
     \frac{d^2 f(x)}{d x_n x_1 } \frac{d^2 f(x)}{d x_n x_2} \dots \frac{d^2 f(x)}{d x_n^2}    
  \end{bmatrix}
\end{align}
\\ \\
\framebox[\linewidth]{
    \begin{minipage}{\dimexpr\linewidth-2\fboxsep-2\fboxrule\relax}
        \textbf{Note:} Diagonal matrices' eigenvalues are the values along the diagonal.
    \end{minipage}
}

\subsubsection{Examples}
\begin{itemize}
  \item The 1D function $f(x) = \frac{1}{2}x^2$ is a convex function because its hessian is $\nabla^2 f(x) = 1$. Since this is a positive value and this matrix is positive semidefinite, the function $f$ is convex. 
  \item The 1D function $f(x) = x^3$ has a hessian of $\nabla^2 f(x) = 6x$. This function is convex on $\mathbb{R}_{++}$ and concave otherwise (both at x =0).
  \item The 1D function $f(x) = \frac{1}{x}$ is convex on the domain $\mathbb{R_{++}}$. The hessian of this function is $\nabla^2 f(x) = \frac{2}{x^3}$, which is positive on $\mathbb{R_{++}}$
  \item The 1D function $f(x) = -log(x)$ is convex on the domain $\mathbb{R_{++}}$. The hessian of this function is $\nabla^2 f(x) = \frac{1}{x^2}$
\end{itemize}

\subsubsection{Linear regression}
Attempting to verify if the least squares equation is convex.
\begin{equation}
  \min_x \frac{1}{2n} \sum_{i=1}^n (a_i^T x - b_i)^2
\end{equation}
\begin{gather}
  \nabla f(x) = \frac{1}{2n} \sum_{i=1}^n \frac{d }{d x} (a_i^T x -b_i)^2 \\
  = \frac{1}{2n} \sum_{i=1}^n (a_i^T x -b_i)a_i \\
  \nabla^2 f(x) = \frac{1}{2n} \sum_{i=1}^n \frac{d }{d x} a_i^T x a_i - b_i a_i \\
  = \frac{1}{2n} \sum_{i=1}^n a_i^T a_i
\end{gather}
This is convex since $a_i^T a_i$ is a symmetric positive semi-definite matrix.

\section{Wednesday 10/02/2024}
\subsection{Convex Functions}
\subsubsection{Verifying convexity}
\begin{itemize}
  \item $\frac{1}{2n} \sum_{i=1}^n (a_i^T x - b_i)^2$
  \item $a^T x + b$
  \item $-\log(x)$
  \item $\frac{1}{x}, x>0$
  \item $\frac{1}{x}, x \neq 0$
  \item $\frac{1}{x}, x<0$
  \item $g(x) + h(x)$
  \item $\frac{g(x)}{h(x)}$
  \item $g(h(x))$
  \item $\max\{f_1(x),f_2(x)\}$
\end{itemize}
\subsubsection{composition rule for convexity}
A logical conclusion for a composition rule $f(x) = g(h(x))$ would be that $f$ is convex if both $g$ and $h$ are convex. However, this does not work because we can use the example $g(x) = -x$ and $h(x) = -\log(x)$. The composite would be $f(x) = \log(x)$ which is concave, not convex. Therefore in order to construct a set of rules for establishing convexity, we can take the second derivate of a composite
\begin{gather}
  f(x) = g(h(x)) \\ 
  \nabla f(x) = h^\prime(x) g^\prime(h(x)) \\
  \nabla^2 f(x) = h^{\prime \prime}(x) g^\prime(h(x)) + h^\prime(x)^2 g^{\prime \prime}(h(x))
\end{gather}
$g$ needs to be convex, and either $h$ is convex and $g$ is non-decreasing or $h$ is concave and $g$ is non-increasing.

\subsection{Convex Sets}
If a function $f$ is convex, then the set
\begin{equation}
  \Omega = \{ x | f(x) \leq c \}
\end{equation}
is convex. The notation above means "The set of all $x$ given that a function $f(x)$ is less than $c$"
\begin{gather}
  \lambda f(x_1) + (1-\lambda) f(x_2) \geq f(\lambda x_1 + (1-\lambda)x_2) \\
  \text{since } f(x_1), f(x_2) \leq c \\
  \lambda c + (1-\lambda) c \geq f(\lambda x_1 + (1-\lambda)x_2) \\ 
  c \geq f(\lambda x_1 + (1-\lambda)x_2)
\end{gather}
The last line implies that any point on the line $f$ that is between two arbitrary points $x_1,x_2 \in \Omega$ is less than $c$. \\
What about the set 
\begin{equation}
  \hat{\Omega} = \{x | f(x) = c\} 
\end{equation}
This set is not convex. Equality constraints are convex if $f$ is a linear function.